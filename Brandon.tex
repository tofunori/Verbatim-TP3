
\begin{minipage}{\textwidth}
\colorbox{gray!10}{
\begin{minipage}{0.98\textwidth}
\small\textbf{Fiche technique de l'entretien}

\begin{tabular}{ll}
\textbf{Participant:} & Homme, 30 ans, niveau d'études universitaire \\
\textbf{Lieu de résidence:} & Trois-Rivières (originaire de Louisville), quartier des Forges \\
\textbf{Durée de résidence:} & 17 ans à Trois-Rivières, 4 ans dans le logement actuel \\
\textbf{Type d'habitation:} & Maison unifamiliale \\
\textbf{Date de l'entretien:} & 2025-03-06 \\
\textbf{Durée:} & 15 minutes 33 secondes \\
\end{tabular}
\end{minipage}}
\end{minipage}

\vspace{1em}

\textbf{Thierry :} Bonjour, mon nom est Thierry. Aujourd'hui, je te pose un guide d'entretien semi-dirigé. Le projet est pour le département de l'environnement dans le cadre du cours analyse du paysage. On interroge des participants en réponse aux attitudes face au changement climatique. Puis je te demande de répondre le plus sérieusement possible en développant le plus possible tes questions. Il n'y a pas de bonne ou mauvaise réponse, mais l'entrevue devrait durer environ une quinzaine de minutes. Puis je te demande d'être le plus sérieux possible. Donc, est-ce que tu es prêt?

\textbf{Participant :} Oui, je suis prêt.

\textbf{Thierry :} OK. Bon. Donc, toi, tu viens de Trois-Rivières?

\textbf{Participant :} Exactement, euh... Pas de Trois-Rivières, mais Louisville, mais déménagé à Trois-Rivières.

\textbf{Thierry :} Exactement. Tu as quel âge?

\textbf{Participant :} 30 ans.

\textbf{Thierry :} Ton niveau d'études le plus haut atteint, c'est soit primaire, secondaire, collégiale ou universitaire?

\textbf{Participant :} Universitaire.

\textbf{Thierry :} Parfait. Puis, ça fait combien de temps que tu habites à Trois-Rivières?

\textbf{Participant :} Humm....Ça doit faire 17 ans.

\textbf{Thierry :} 17 ans, OK. Puis, ça fait combien de temps que tu habites ou que tu habites présentement?

\textbf{Participant :} Quatre ans.

\textbf{Thierry :} Puis, c'est quel quartier que tu habites?

\textbf{Participant :} Le nom du quartier? Je ne sais pas. Je pense que ça doit être des Forges.

\textbf{Thierry :} ok, puis c'est quel type d'habitation que tu habites?

\textbf{Participant :} Maison unifamiliale.

\textbf{Thierry :} Parfait. On va commencer le questionnaire. Ma première question, c'est pour toi, qu'est-ce que ça signifie, le changement climatique?

\textbf{Participant :} Je réponds, là? 

\textbf{Thierry :} Oui

\textbf{Participant :}Le changement climatique, c'est les variations pour moi des températures puis des changements qui se produisent au niveau de la planète. On parle souvent de la température qui peut être soit plus chaude ou la fonte des glaciers. Mais ouin, c'est ça. C'est les changements qui se produisent au niveau de la planète, au niveau de l'environnement.

\textbf{Thierry :} ok...donc pour toi, les changements climatiques proviennent d'où, en fait?

\textbf{Participant :} Ben...je pense qu'il y a un facteur naturel qui fait en sorte... Je ne suis pas moi l'expert, mais je pense qu'il y a un facteur qui est naturel qui vient avec des cycles. Je pense qu'il y a une partie que ça peut être ça, puis l'autre partie, ça peut être la pollution, puis tout ce que nous, comme êtres humains, on peut faire de néfaste à l'environnement.

\textbf{Thierry :} OK. Puis, à ce moment-là, si c'est, comme tu dis, c'est les humains qui, si je suis bien d'accord avec ce que tu dis, qui causent les changements climatiques, c'est comment les résoudre ces changements climatiques-là, selon toi?

\textbf{Participant :} Euh..Ça peut être de faire une consommation plus responsable, favoriser des produits qui vont dégager moins de gaz à effet de serre. Ça peut être toute la question de la consommation responsable. Pour les entreprises, favoriser un développement durable, un approvisionnement plus responsable aussi, d'avoir des mesures éthiques dans leur code de conduite, dans leur façon de procéder. Puis je pense que c'est un ensemble de tout le monde qui doit faire attention à ce qu'on consomme et la manière qu'on agit.

\textbf{Thierry :}OK... Dans le fond, selon toi, c'est quoi les impacts des changements climatiques dans ton quotidien à toi?

\textbf{Participant :} Euh...Dans mon quotidien, moi?
 
\textbf{Thierry :} Est-ce que t'en vois?
 
\textbf{Participant :} Bien, bien, c'est sûr que si je compare par rapport à ce que j'avais 10 ans par rapport à ce que j'ai 30 ans, je vois pas vraiment de différence marquée dans ma vie personnelle. Non, je dirais qu'en ce moment, c'est sûr qu'il y a sûrement une façon que ça m'affecte, mais que je n'en suis juste pas conscient. Mais au quotidien, je ne me dis pas que c'est à cause des changements climatiques. Je n'ai pas rien qui me vient en tête.

\textbf{Thierry :}Puis ok... Il n'y a pas d'expérience nécessairement de chaleur, des vagues de chaleur, etc., que tu pourrais associer à des changements climatiques?

\textbf{Participant :} Ben tsé...À chaque année, on pourrait dire aussi des deux côtés qu'il y a eu des tempêtes. On pourrait avoir des records de tempêtes de neige, mais à quel point c'est un effet du réchauffement climatique ou c'est juste qu'à un moment donné, on est fait pour battre un record de chaleur ou un record de tempête de neige. Je ne sais pas à quel point c'est nécessairement associé directement au changement climatique. C'est dur pour quelqu'un qui n'est pas dans le domaine de relier ça directement à ça.

\textbf{Thierry :} Justement, comment tu décris ton expérience de la chaleur dans ton quartier?

\textbf{Participant :} Euh...Ben, relativement bien (rires). C'est sûr que l'été, quand il fait chaud, nous, on est à l'air climatisé, on n'a pas vraiment l'effet de chaleur. Je ne pense pas que c'est... Je ne trouve pas que c'est nécessairement pire qu'il y a 5 ans ou 10 ans. C'est comme un état que je suis habitué. Je ne pourrais pas dire que je ne relie pas la chaleur de l'été au réchauffement climatique. Pour moi, c'est juste l'été, c'est normal des fois qu'il fait 35-40 et qu'il y a de l'humidité.

\textbf{Thierry :} Y a-t-il une journée en particulier où tu as ressenti la chaleur vraiment forte que tu te rappelles?

\textbf{Participant :} Non, pas particulièrement.

\textbf{Thierry :} Par exemple, comment tu te sens quand tu penses aux effets de la chaleur à plus grande échelle ou sur le long terme, par exemple?

\textbf{Participant :}Humm... C'est sûr que je pense que ça a des effets néfastes. Je pense qu'éventuellement, il va falloir que quelqu'un... Ben...Je sais qu'il y a déjà quelqu'un qui se penche sur la question et qui va essayer de trouver des solutions. Mais oui, si je regarde ma fille, je veux qu'elle soit capable de vivre dans un environnement qui va être tolérable. Donc, oui, c'est ça. Comment je me sens, je dirais que je suis quand même relativement assez neutre, même si je le sais que c'est une question qui peut être importante.

\textbf{Thierry :} Par exemple, c'est où que les endroits que tu te sens le plus à l'aise pendant qu'il y a des vagues de chaleur? Puis pourquoi?

\textbf{Participant :} Ben, chez nous. Moi, je suis vrai. Chez nous, j'ai ma thermopompe. Quand il fait chaud, des fois, je peux aller dehors, il fait chaud, puis après ça, je rentre en dedans. On est super bien à l'air-clim. C'est sûr que quand il fait très chaud, c'est probablement chez nous que je suis le mieux.

\textbf{Thierry :} Par exemple, quand il y a des vagues de chaleur, comment ça l'influence ta manière d'utiliser ton quartier ou même la ville?

\textbf{Participant :} J'ai potentiellement plus le goût d'utiliser mon auto. Soit je ne sors pas, mais si j'ai à sortir, je vais favoriser plus mon auto parce qu'il y a l'air climatisé versus aller courir dehors ou faire du vélo. Je dirais que soit j'utilise moins le transport et je sors moins, ou si j'ai à l'utiliser, je vais utiliser juste plus l'auto.

\textbf{Thierry :} Comment tu réagis quand il y a des journées vraiment chaudes? Comment tu te sens?

\textbf{Participant :} Ben...Honnêtement, ça ne m'affecte pas tant parce que mon travail est de la maison. S'il fait vraiment chaud, je vais juste rester dans la maison à l'air climatisé. Je n'ai pas vraiment l'impression qu'il fait très, très chaud. Donc, c'est certain peut-être que je suis une exception, mais la chaleur, ça m'affecte peut-être moins que quelqu'un qui travaille dans la construction, par exemple.

\textbf{Thierry :} Humm, humm..Mais, tu sais, y a-tu comme, t'as-tu justement pour ça modifié des parties de ton logement ou ton environnement qui fait que tu supportes mieux la chaleur que d'autres?

\textbf{Participant :} Humm...Non, mais c'est sûr que quand on a acheté la maison, il n'y avait pas d'air climatisé. C'était une des premières choses qu'on a fait installer. Mais à part avoir fait installer un air climatisé, il n'y a pas d'autre chose qu'on a modifié concrètement.

\textbf{Thierry :} ok, puis par exemple, quand il fait vraiment chaud, il y a-t-il des pratiques que tu adoptes, par exemple, pour mieux gérer la chaleur? Est-ce que tu vas avoir tendance à aller te baigner? Il y-tu justement des choses que tu modifies?

\textbf{Participant :} Non, mais c'est sûr que pendant l'été, je sais qu'une journée dans la semaine ou deux journées qui va faire beaucoup chaud, je vais modifier ma semaine d'entraînement parce que je m'entraîne beaucoup. Je fais beaucoup de courses, vélo. C'est sûr que souvent, si je le sais que, par exemple, mardi, il va faire vraiment chaud, je vais m'assurer que mardi, c'est peut-être ma journée de repos, puis que je vais faire mes grosses sorties de course ou de vélo une autre journée. C'est pas mal ça que je modifie. Non, pas vraiment d'autres choses.

\textbf{Thierry :} Par exemple, comment tu vois l'évolution de ton quartier si les températures continuaient à augmenter? 

\textbf{Participant :} À quel terme? Dans 100 ans ou dans 5 ans? C'est sûr que dans 5 ans, je ne pense pas qu'il va y avoir un gros déchangement. Je pense que le quartier va continuer à se développer. Il est en développement en ce moment.

\textbf{Thierry :} Justement, ce développement-là, est-ce que ça pourrait affecter comment tu vis la chaleur durant la période estivale? Si il y a de plus en plus de maisons qui se développent, moins en moins d'espace vert.

\textbf{Participant :} Non, parce qu'il y a quand même pas mal d'espace vert. Je ne pense pas. Ça fait quoi qu'il y ait plus d'espace vert pour la chaleur?

\textbf{Thierry :} Ça diminue, ça l'absorbe la chaleur. Plutôt, ça la réfléchit au lieu de l'absorbe la chaleur plutôt ça la réfléchit au lieu de l'absorber. L'asphalte va absorber la chaleur puis va l'émettre après. 

\textbf{Participant :}Je ne pense pas que ça fasse une différence assez significative pour que moi je vois une différence à moyen terme. Peut-être dans 50 ans, oui, mais dans 50 ans, je vais achever de toute façon.

\textbf{Thierry :} Donc, il faudrait que ton quartier se développe énormément, peut-être, pour que tu puisses... 

\textbf{Participant :} Oui, oui, oui. Il faudrait que ça soit style Montréal avec des gros buildings, puis je ne pense pas que ça soit ça.

\textbf{Thierry :} Mais sinon, mettons, qu'est-ce que tu penses qui pourrait être fait dans ton quartier pour que ça soit encore plus vivable les périodes de chaleur?

\textbf{Participant :} Euh... Je pense que dans notre quartier, il manque un parc. Ce serait le fun d'avoir un parc. Moi, je ne suis pas piscine, mais tsé...quand je vais, des fois, on va au parc Lambert qui est à 10 minutes d'auto de chez nous. Il y a parc, piscine, les enfants jouent, mais dans notre coin, il y a juste un parc. Il est grand le parc, mais il y a juste un module et il n'y a rien d'autre. Des fois, quand il y a des chaleurs, je pense que d'avoir un parc avec des arbres, des piscines, des jeux d'eau, ça serait cool.

\textbf{Thierry :} Oui, je comprends. Puis, comment tu vois justement les changements climatiques à long terme?

\textbf{Participant :}(hésitation) Je pense que ça risque de continuer d'empirer. Mais en même temps, non, c'est ça, ça risque de continuer d'empirer. Sûrement que de plus en plus, il va y avoir des mesures qui vont être prises pour améliorer. Mais je pense que le fait que la population va continuer à grossir, de plus en plus, il va y avoir du développement économique. Les mesures positives qui vont être mises, je pense qu'elles vont quand même être contrebalancées par le développement économique et le développement de la société en général. Je pense que ça va faire un effet net que la pollution va continuer à augmenter et que ça risque de continuer à se détériorer. Je pense que c'est comme ça que je vois ça.

\textbf{Thierry :} Donc, tu vois peut-être qu'en lien avec les changements climatiques, les vagues de chaleur pourraient continuer à augmenter en ville quoi?

\textbf{Participant :} Probablement. Je ne suis pas un expert, mais en tant que seulement un petit kinésiologue, je pense que ce serait la logique que je pourrais suivre.

\textbf{Thierry :} C'est bon. Écoute, ça fait le tour de nos questions. Donc, merci pour tes réponses, puis à la prochaine. 

\textbf{Participant :} Merci Thierry.
