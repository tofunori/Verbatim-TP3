
\begin{minipage}{\textwidth}
\colorbox{gray!10}{
\begin{minipage}{0.98\textwidth}
\small\textbf{Fiche technique de l'entretien}

\begin{tabular}{ll}
\textbf{Participante:} & Femme, 65 ans, niveau d'études secondaire 2 \\
\textbf{Lieu de résidence:} & Trois-Rivières, centre-ville \\
\textbf{Durée de résidence:} & 17 ans à Trois-Rivières, 5 ans dans le logement actuel \\
\textbf{Type d'habitation:} & Appartement en immeuble \\
\textbf{Date de l'entretien:} & 2025-03-05 \\
\textbf{Durée:} & 17 minutes 15 secondes \\
\end{tabular}
\end{minipage}}
\end{minipage}

\vspace{1em}

\textbf{Thierry :} Bonjour, mon nom est Thierry, et dans le cadre du cours d'analyse du paysage, on fait une entrevue semi-dirigée pour voir la latitude de la population face aux vagues de chaleur et aux changements climatiques. Donc, ici, il n'y a pas de bonne ou mauvaise réponse. Tu peux prendre le temps de répondre, puis si tu as des questions, n'hésite pas à me demander. Donc, on va commencer comme ça. Est-ce que ça te va?

\textbf{Participante :} Oui.

\textbf{Thierry :} Parfait. Bon, tout d'abord, pour toi, qu'est-ce que ça signifie, le changement climatique ?

\textbf{Participante :} Bien, moi, quand j'entends parler de changement climatique, oui, j'y crois. Des fois, je n'y crois pas non plus. Sauf que, c'est sûr, quand je vois toutes les inondations, les feux de forêt, je m'interroge. Je sais qu'on a beaucoup à faire pour changer tout ça au niveau de nos coutumes de vie.

\textbf{Thierry :} D'où ça provient, tu penses, le changement climatique ?

\textbf{Participante :} C'est plutôt justement toute la pollution qu'on met dans l'air. Moi, je suis sûre que je... Moi, je vois plus les grosses compagnies qui nous polluent à longueur de journée. Je veux dire, si je parle de moi, ma petite vie bien simple, je ne penserais pas que c'est moi qui pollue le plus.

\textbf{Thierry :} Comment les résoudre ces changements climatiques-là ?

\textbf{Participante :} Bien? C'est ça. C'est en changeant nos habitudes. C'est sûr que maintenant, on est beaucoup plus sur la terre qu'avant. On consomme et on ne regarde pas trop trop ce qu'on fait. Tout est en rapport avec l'argent, les industries. On crée des choses, mais ça ne veut pas dire que c'est les bonnes choses. Ça, là-dessus, je ne veux pas dire qu'on tomberait dans l'ancien temps où tout était pas mal minime. L'air était pas mal plus pur aussi. C'était quand même une bonne chose, mais c'est sûr que ce n'est pas moi qui vais changer le monde demain matin, mais c'est sûr que c'est pas moi qui va changer le monde demain matin mais c'est les grandes industries qui eux il faudrait vraiment qu'ils se posent des questions à faire tous ces changements-là. Je veux dire nos bouteilles d'eau, tout ce qui est jeté dans le fleuve, l'eau est polluée.

\textbf{Thierry :} Justement, pour toi c'est quoi l'impact des changements climatiques dans ton quotidien ? T'en vois-tu ?

\textbf{Participante :} Moi, l'impact... Est-ce que tu parles juste de moi? De moi ou dans... 

\textbf{Thierry :} Non, pour toi, vois-tu au quotidien les impacts des changements climatiques ?

\textbf{Participante :} Moi, je pourrais dire non. Moi, au quotidien, je veux dire, je suis une fille qui marche. Je ne pense pas que l'air, je ne la sens pas polluée. Je suis une fille en santé. À part que les grosses chaleurs l'été, des fois, qu'à salarier, mais je pense qu'il faisait chaud.... J'ai connu il y a longtemps, j'ai quand même 60 quelques années, là, fait que j'ai connu il y a bien longtemps, on en a eu aussi des chaleurs, fait que, je veux dire, ça dure pas. Je sais si ça durerait tout le temps. La seule chose que des fois, je remarque, c'est qu'avant, on avait le printemps, surtout, là, il était plus, il arrivait plus vite, tu sais, au mois d'avril, mes enfants, quand c'était leur fête, déjà, on était habillés vraiment au printemps. Maintenant, c'est vraiment plus long. Sur ça, je peux vraiment remarquer que... 

\textbf{Thierry :} Ça, c'est les changements que tu as remarqués au cours des dernières années.

\textbf{Participante :} Oui, ça, c'est frappant. C'est comme si on a un décalage au niveau de... Mais par contre, l'automne, des fois, elle va loin aussi, puis le printemps.

\textbf{Thierry :} Mais quand il arrive des chaleurs, comment tu te sens ?

\textbf{Participante :} Quand il arrive des chaleurs? Je veux dire, moi, j'ai toujours bien enduré quand même la chaleur. Je veux dire, je sais que ce n'est pas pour longtemps. Je la subis quand elle arrive. J'ai toujours hâte qu'on en finisse. Mais en même temps, je pense qu'on en a toujours eu des chaleurs quand même. Je veux dire c'est pas nouveau. Je ne suis pas née de la dernière pluie. J'en ai vu des chaleurs. Ça fait un temps et ça l'arrête. Mais moi, ça ne me préoccupe pas. Pour le moment, je ne suis pas encore préoccupée par les chaleurs d'été, admettons.

\textbf{Thierry :} Y a-t-il des situations qui sont liées aux vagues de chaleur que tu trouves particulièrement agréables ou difficiles ?

\textbf{Participante :} Moi, je n'ai jamais aimé les grosses chaleurs. Je trouve ça vraiment désagréable. Mais il y a aussi, qu'est-ce qu'il faut dire? C'est que je pense qu'on est un peu complices de tout ça pareil parce que toutes les airs climatiques qu'on s'installe dans nos maisons, ça aussi, ça projette de la chaleur, pareil. Puis, c'est pas bon pour ça, dans l'air tsé la... c'est ce que j'avais entendu dire. Il y a trop d'autos, il y a trop d'airs....tsé Ça, l'air climatisée, des fois, on n'y pense pas, mais c'est toutes ces choses-là qui font que la planète change.

\textbf{Thierry :} Mais il n'y a pas de situation liée à la chaleur que tu trouves nécessairement difficile l'été? Quand il fait chaud, par exemple? 

\textbf{Participante :}À venir jusqu'à date, je suis capable de la supporter. Mais comme je t'ai dit, elle ne dure pas six mois. Je sais qu'elle fait quelques jours. C'est juste que cette année, je me suis acheté un petit climatiseur, mais je ne sais même pas si je vais le faire fonctionner. Je n'aime vraiment pas ça. Je trouve des moyens. On ferme les rideaux le jour. Moi, ça me va encore. Je ne suis pas traumatisée et je ne suis pas anxieuse face à ça, les changements climatiques. Parce que des fois, les médias en mettent beaucoup aussi là-dessus.

\textbf{Thierry :} Comment tu te sens, par exemple, si tu penses aux vagues de chaleur, aux effets de la chaleur sur le long terme ?

\textbf{Participante :} Bien, moi, je ne pense pas que sur le long... Tu veux dire à long terme, dans les années futures, pour les chaleurs? 

\textbf{Thierry :} Si jamais les vagues de chaleur augmentent, comment tu te sens par rapport à ça? 

\textbf{Participante :} Bien, moi, je vis au jour le jour. Je me dis, on verra dans le temps. Ça ne donne rien de précipité. La nature est tellement changeante que peut-être cet été, ça ne va pas être si pire. Comme je t'ai dit, je ne suis pas anxieuse. Je ne le précipite pas. Je ne le vois pas de façon négative. Je ne suis pas négative là-dedans. Non, non.  Parce que moi, excuse-moi, mais s'il n'y en mettait pas tant, mettons, il n'y aurait plus de télévision, je trouverais que la vie est correcte. Je ne manque pas d'air. Mais des fois, les médias, ils nous en mettent tellement pire que c'est que je pense que c'est ça qui fait que les gens sont anxieux. J'essaie de ne pas trop écouter ça et voir que la vie est belle encore pareil.

\textbf{Thierry :} Mais justement quand ça arrive, ces vagues de chaleur-là, c'est où que tu te sens le plus à l'aise ou le mieux ?

\textbf{Participante :} Dans mon auto à l'air climatisé (rires). Pour vrai, quand il arrive des chaleurs comme ça, moi, j'aime mieux pas sortir. Moi, je suis bien... Si j'ai pas affaire à sortir, là, je vais garder mon logement tempéré, là, à... Tu sais, le... Quand il fait soleil dans les fenêtres, je vais fermer les rideaux à ces endroits-là. Mais... Puis moi, je marche beaucoup, fait que je vais plus marcher après souper. C'est comme je vous dis, par chance que ça dure pas longtemps, longtemps, parce que sinon, c'est quand même pas le fun de se priver de sortir, de faire des activités quand il fait vraiment chaud comme ça. Mais il y a toujours moyen de faire des choses tôt le matin, puis laisser passer ça, puis repartir le soir en vélo, quelque chose. C'est juste que c'est comme une adaptation quand il arrive ces périodes-là. C'est comme une tempête de neige. Je veux dire, on ne fait pas exprès pour prendre la rue. Moi, je m'adapte à chaque situation. Il n'y a rien d'alarmant quand on y pense tsé...

\textbf{Thierry :} Il y a-t-il des endroits dans ton quartier que tu  aimes utiliser ou qui vont influencer la manière dont tu utilises ton quartier quand il y a des vagues de chaleur ?

\textbf{Participante :} Oui, dans mon quartier, c'est plein de magasins. Je vais peut-être me dire, je sais qu'il annonce une vague de chaleur. Aujourd'hui, on est encore correct. Je vais faire mes activités extérieures. Demain, j'irai faire l'épicerie. Ça adonne bien. J'avais des commissions à faire. Ils disent que tu as au moins deux heures au frais dans une journée pour que ton système fonctionne bien.

\textbf{Thierry :} Vas-tu avoir tendance à utiliser, par exemple, les rues plus ombragées avec des arbres ?

\textbf{Participante :} Ça, oui. Ça va être plus comme le matin, aller au parc avec plein d'arbres. Je suis vraiment bien. Quand je sens que mon activité est faite, genre un 10-12 kilomètres de marche, là je sais que chez nous, je suis bien. Je ne fais pas exprès pour m'exciter les sangs, comme on dit. On va prendre ça relax. Je me dis que c'est l'été, donc on prend ça.... On attend que ça passe... On fait d'autres choses pareilles entre-temps. Je ne suis pas en colère. Je ne suis pas négative du tout face à ça. Il y a plein d'autres choses. Quelqu'un qui a de l'imagination, il peut faire plein de choses en attendant que ça passe.

\textbf{Thierry :} Donc par exemple dans ton logement, y a-t-il des choses que tu as modifiées pour mieux supporter la chaleur pendant l'été ? As-tu installé des climatiseurs, des ventilateurs ?

\textbf{Participante :} Bien, ici, oui.

\textbf{Thierry :} Souffres-tu de la chaleur dans ton appartement ?

\textbf{Participante :}Moi, j'ai jamais souffert de la chaleur ici parce que j'ai un arbre en avant sur mon perron. Puis lui, il me projette beaucoup d'ombre. Puis ça fait que je n'ai pas, par contre, de perron. Tu sais, je n'ai un en arrière, mais je n'ai pas un en avant. Mais par contre, le perron, que je vous dis qu'il y a un arbre, ça, ça projette beaucoup d'ombre. Puis en plus, comme à 5 heures, je peux vraiment aller m'asseoir. Il n'y a plus de soleil, même 4 heures. Puis je peux passer la soirée sur le perron. Je suis très, très bien, sans air climatisé dans la maison. Moi, je l'ai acheté juste en fin de l'été, mon air climatisé. Mais honnêtement, j'ai toujours été capable de tolérer cette chaleur-là, juste en ouvrant mes fenêtres durant toute la nuit, le soir au coucher, pour faire rentrer l'air. Puis le jour, quand je vais travailler, je ferme les rideaux, puis j'arrive en après-midi, puis mon logement, il reste quand même frais, mais je fais comme pas exprès. J'essaie de prendre des conseils qui fonctionnent, sans exagérer sur toutes ces affaires-là d'air climatisé. Mais tu sais, je sais qu'il y en a qui n'ont pas le choix parce qu'ils n'ont pas de santé, puis ils doivent être climatité comme ça, à l'air climatisée. Mais je veux dire, quelqu'un en santé est capable de passer au travers. Assez, oui. 

\textbf{Thierry :} Puis, ok. Comment, par exemple, tu imagines l'avenir dans ton quartier si les températures continuent à augmenter, par exemple l'été ? Tu penses que ça deviendrait pire, ça deviendrait insupportable ?

\textbf{Participante :} C'est sûr qu'ici, c'est des blocs. On n'a pas d'espace vert vraiment. C'est plein de magasins. C'est sûr que la chaleur, c'est comme étouffant, comme on dit, là. Tu sais, tu sors dehors, t'étouffes, là. On s'en cache pas de le dire, mais on sait que c'est l'été. Et puis, si ça rampait avec les années, bien là, je sais pas, là. Je veux dire...

\textbf{Thierry :} Qu'est-ce qui pourrait être amélioré en ville, justement, dans ton quartier pour améliorer ou diminuer l'impact de la chaleur durant les mois ?

\textbf{Participante :} Bien, moi, si l'impact de la chaleur est ici dans mon quartier, je dis probablement qu'elle est partout aussi. Je veux dire, c'est dans l'air, la chaleur. Je ne pense pas que ça soit juste dans mon quartier.

\textbf{Thierry :} Y aurait-tu, par exemple, des installations que tu verrais dans ton quartier qui pourraient améliorer ça ?

\textbf{Participante :} Ça serait bien, oui, mais j'ai du mal à l'imaginer parce que toutes les maisons sont toutes collées. C'est plein de magasins. Je veux dire, je suis vraiment... C'est central, moi, où est-ce que j'habite. Moi, j'ai un arbre. Je suis contente. On en a un petit peu, un peu partout, mais c'est suffisant pour voir un impact. Même s'il en mettrait plus, je ne penserais pas. En ville, ça reste la ville. Puis la campagne, à l'air frais, ça reste la campagne. 

\textbf{Thierry :}Donc,, ça serait pas assez d'implanter des arbres pour vraiment diminuer les îlots de chaleur ou les vagues de chaleur? 

\textbf{Participante :}Je sais pas où est-ce qu'ils mettraient les arbres. C'est ça, le problème. Dans mon quartier, en tout cas, quand c'est urbain, quand c'est en ville, c'est très difficile de faire quoi que ce soit. Soit qu'on la subisse ou on change notre manière de se comporter. Je ne sais pas.

OK. Bon, fait que... Mais la plupart, partout où tu travailles, dans les bureaux, ils ont tout l'air climatisé. Fait qu'eux autres, je pense que ça ne leur empêche pas de travailler pareil. Puis, OK,es industries qui doivent changer leur comportement. Eux autres, c'est de l'argent qu'ils veulent. Ils veulent toujours le profit, le profit, mais c'est au détriment de notre santé pareil. Même les avions, tout ça, ça circule. Moi, personnellement, je ne pense pas que je suis la personne qui détruit la planète en ce moment. Je vis de simplicité volontaire. Si tout le monde m'aimait comme moi, vivre de la manière que moi je vis, je pense qu'on n'aurait pas de problème. Je sais que ça prend des industries et qu'il faut fabriquer plein de choses. Peut-être en avoir moins. Je trouve qu'on en a pas mal des fois pour rien des industries.

\textbf{Thierry :} Toi, tu as quel âge ?

\textbf{Participante :} Je vais avoir 65 ans.

\textbf{Thierry :} Ton niveau d'études le plus haut atteint ?

\textbf{Participante :} Secondaire 2 parce que j'ai dû aider ma mère qui était malade.

\textbf{Thierry :} Ça fait combien de temps que tu habites ici ?

\textbf{Participante :} À Trois-Rivières, 17 ans. Dans ton appartement ici? 5 ans.

\textbf{Thierry :} Je pense que ça fait le tour. Merci beaucoup pour tes réponses. C'était très bien répondu. Pas très bien répondu, mais c'était très intéressant. Merci. Merci beaucoup. Merci.
